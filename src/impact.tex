\section{Impact}
Assigned: Avery Clotfelter

Autonomous satellite swarm technology is essential to the future of
advanced space missions.  Formation flight of small spacecraft offers
superior resilience and versatility as well as increased mission
capability at significantly lower cost.  One of the primary weaknesses
of current formation flight missions is the need for satellite
instructions to be delivered and assessed from the ground, a process
that is resource intensive and becomes detrimental as missions take
place further from Earth.  With on-board trajectory planning and real-time control algorithms, large swarms of satellites can navigate autonomously with respect to a target orbit, thereby eliminating ground station communication challenges.  This project will be the first demonstration that such algorithms are feasible in practice.  NASA has exhibited strong interest in the
following technologies which are likely of primary concern to the DoD
as well: collaborative mobility, perception for autonomous systems,
fault diagnosis and correction, swarm navigation, and distributed
aperture.  Distributed aperture can play a strong role in improving
defense and surveillance through higher resolution, cheaper, and more
resilient Earth imaging systems.  Autonomous research on CubeSat
swarms currently being conducted at the University of Vermont overlaps
with several established institutions such as JPL, the Armstrong
Flight Research Center, and the Kennedy Space Center. 

Vermont CubeSat swarm research also serves to increase Vermont’s small
footprint in the space exploration industry.  By way of joint research
capacity between UVM and VTC, with support from space mobility partner
Benchmark Space Systems, the project seeks to grow Vermont’s presence
in satellite research.  In addition to impacting Vermont’s engineering
direction as a whole, the project directly influences the education of
students, both high school and university, who are involved in the
project.  Through software and hardware training, educational
resources on orbital mechanics, propulsion, and smallsat development,
the project is setting the stage for a strong Vermont university space
program.  Collaboration between IEEE and AIAA with continuous support
from Benchmark Space Systems will further establish this program.

Several novel technologies are being incorporated into the CubeSat
swarm with the primary goal of expanding the current capabilities of
1U CubeSats.  One of the most impressive accomplishments will be the
inclusion of an inert non-toxic propellant thruster system developed
by Benchmark Space Systems with less than 0.5U volume.  The thruster
will be capable of providing the necessary 20 N-s impulse and 40 m/s
delta-V dictated by the control algorithms while fully integrating
with the rest of the CubeSat.  Such a system is significant, as
propulsion systems are not often seen in CubeSats of such small size. 
This demonstration will prove the possibility of 1U propulsion systems
which will increase individual satellite capabilities from detumble to
station-keeping to orbital maneuvers.  The other major divergence from
traditional CubeSat design is the incorporation of PCB-integrated
magnetorquers which will be traced out on three orthogonal solar cell
boards.  Due to the volume constraint put on the CubeSats by the
propulsion system, the CubeSats must forgo the typical magnetorquer
rods.  Preliminary research suggests PCB-printed boards can be easily
manufactured while providing adequate magnetic dipole moments at
higher efficiencies.  Successful demonstration of PCB magnetorquers
would massively increase 1U CubeSat mission potential as additional
volume offers more space for scientific measurement devices, power
systems, ADCS, GPS and radio, or camera systems.
